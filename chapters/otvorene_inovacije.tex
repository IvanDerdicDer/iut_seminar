\chapter{Otvorene inovacije}

Otvorene inovacije su pristup inoviranju koji uključuje suradnju s drugim
organizacijama, zajednicama, stručnjacima i pojedincima izvan vlastite tvrtke,
kako bi se postigao veći uspjeh u inoviranju. Ovaj pristup se obično koristi za
rješavanje složenih problema ili izazova koji zahtijevaju multidisciplinarni
pristup \citep{openinnovation2023}.

Koncept otvorenih inovacija prvi je put predstavio Henry Chesbrough, koji je
tvrdio da je inovativnost važnija od samog izuma, te da bi tvrtke trebale
otvoriti svoje inovacijske procese kako bi uključile vanjske suradnike. U
otvorenim inovacijama, tvrtke mogu koristiti vanjske izvore znanja, kao što su
akademski istraživački centri, start-up tvrtke, druge tvrtke u industriji ili
sami korisnici proizvoda, kako bi stvorili nove proizvode i usluge.

Ključni aspekti otvorenih inovacija su dijeljenje znanja, transparentnost i
suradnja. Ovaj pristup potiče stvaranje šire mreže suradnje i inovacije, što
može pomoći tvrtkama da stvore bolje proizvode i usluge, poboljšaju procese
proizvodnje i povećaju svoju konkurentsku prednost na tržištu \citep{openinnovation2023}.