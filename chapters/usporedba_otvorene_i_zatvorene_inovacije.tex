\chapter{Usporedba otvorene i zatvorene inovacije} \label{ch:usporedba_otvorene_i_zatvorene_inovacije}

Otvorena inovacija i zatvorena inovacija su dvije različite strategije
upravljanja inovacijama unutar tvrtke. Glavna razlika između ovih pristupa leži
u tome kako se generira inovacija i u kojoj mjeri se uključuje vanjsko znanje u
proces inovacije.

\section{Prednosti i mane otvorenih inovacija} \label{sec:prednosti_i_mane_otvorenih_inovacija}

Otvorena inovacija je suradnički pristup koji omogućuje organizacijama
iskorištavanje vanjskih znanja, vještina i resursa kako bi poboljšale
inovacijske rezultate i otkrile nove ideje (vidi
poglavlje~(\ref{ch:otvorene_inovacije})). Evo nekih prednosti i nedostataka
otvorene inovacije:

\textbf{Prednosti otvorene inovacije}:

\begin{itemize}
\item Proces razvoja inovacija postaje mnogo učinkovitiji i brži \citep{gotechinnovationcompare2021}.
\item Otvorena inovacija može biti ekonomičnija od tradicionalnih metoda \citep{a2dopeninnovation2023}.
\item Otvorena inovacija omogućuje stvarne povratne informacije od raznolikog
skupa ljudi koji nisu ograničeni istim odgovornostima kao zaposlenici unutar
organizacije \citep{a2dopeninnovation2023}.
\item Uključivanje mnoštva omogućava izloženost raznolikom nizu ideja,
prijedloga i perspektiva \citep{a2dopeninnovation2023}.
\item Otvorena inovacija može pomoći organizacijama u identifikaciji novih
talenata \citep{a2dopeninnovation2023}.
\end{itemize}

\textbf{Nedostaci otvorene inovacije:}

\begin{itemize}
\item Otvorenost prema tržištu dovodi do rizika povezanih s procurivanjem
informacija i kibernetičkom sigurnošću \citep{gotechinnovationcompare2021}.
\item Postoje rizici u pogledu donošenja pogrešnog izbora među startupima i
tvrtkama koje nude inovativne proizvode i tehnologije \citep{gotechinnovationcompare2021}.
\item Postoji rizik da talentirani zaposlenici korporativnog tima za inovacije
budu privučeni konkurentskim tvrtkama \citep{gotechinnovationcompare2021}.
\item Pravni aspekti otvorene inovacije često se zanemaruju \citep{timercompare2016}.
\item Otvorena inovacija može se doživljavati kao rizična zbog pitanja
intelektualnog vlasništva \citep{a2dopeninnovation2023}.
\end{itemize}

\section{Prednosti i mane zatvorenih inovacija} \label{sec:prednosti_i_mane_zatvorenih_inovacija}

Zatvorena inovacija, također poznata kao tradicionalna inovacija, odnosi se na
pristup u kojem se nove tehnologije razvijaju s ograničenim korporativnim
resursima i bez vanjske stručnosti (vidi
poglavlje~(\ref{ch:zatvorene_inovacije})). Evo nekih prednosti i nedostataka
zatvorene inovacije:

\textbf{Prednosti zatvorene inovacije:}

\begin{itemize}
\item Korporacija ima potpunu kontrolu nad procesom inovacije i intelektualnim
vlasništvom \citep{asd1}.
\item Korporacija može zadržati povjerljivost poslovnih tajni \citep{asd1}.
\item Korporacija može iskoristiti postojeće resurse i znanje za stvaranje novih
proizvoda \citep{asd2}.
\end{itemize}

\textbf{Nedostaci zatvorene inovacije:}

\begin{itemize}
\item Zatvorena inovacija može rezultirati nedostatkom kreativnosti i inovacije
zbog ograničenog izvora stručnosti i znanja \citep{asd1}.
\item Zatvorena inovacija može biti skupa, jer korporacija mora ulagati u
istraživanje i razvoj te možda ne može iskoristiti vanjske resurse i znanje \citep{asd1}.
\item Korporacija može propustiti vrijedne ideje i povratne informacije iz
vanjskih izvora \citep{asd1}.
\item Zatvorena inovacija može biti spora, jer korporacija se mora osloniti na
vlastite resurse i znanje za razvoj novih proizvoda \citep{asd2}.
\item Korporacija možda neće moći pratiti promjene u trendovima na tržištu \citep{asd2}.
\end{itemize}

\section{Usporedba} \label{sec:usporedba}

Otvorena inovacija i zatvorena inovacija su dvije različite pristupe upravljanju
inovacijama. Zatvorena inovacija se temelji na gledištu da inovacije razvijaju
same tvrtke, pri čemu se inovacijski proces odvija isključivo unutar tvrtke. S
druge strane, otvorena inovacija uključuje vanjska znanja u upravljanje
inovacijama i znači otvaranje inovacijskog procesa izvan granica tvrtke radi
povećanja vlastitog inovacijskog potencijala kroz aktivno strateško korištenje
okruženja \citep{asd3}.

Evo nekih usporedbi između otvorene inovacije i zatvorene inovacije:

\textbf{Proces inovacija:}

U zatvorenoj inovaciji inovacija se razvija u zatvorenom okruženju tvrtke, dok
otvorena inovacija uključuje vanjska znanja u upravljanje inovacijama \citep{asd3}.

\textbf{Kontrola:}

U zatvorenoj inovaciji korporacija ima potpunu kontrolu nad inovacijskim
procesom i intelektualnim vlasništvom, dok u otvorenoj inovaciji korporacija
mora dijeliti kontrolu s vanjskim akterima poput kupaca, dobavljača, sveučilišta
ili drugih tvrtki \citep{asd3}.

\textbf{Kreativnost i inovacija:}

Zatvorena inovacija može rezultirati nedostatkom kreativnosti i inovacije zbog
ograničenog izvora stručnosti i znanja, dok otvorena inovacija omogućuje stvarne
povratne informacije od raznolikog skupa ljudi koji nisu ograničeni istim
odgovornostima kao zaposlenici unutar organizacije \citep{asd3}.

\textbf{Trošak:}

Otvorena inovacija može biti ekonomičnija od zatvorene inovacije jer omogućuje
organizacijama da iskoriste vanjska znanja, vještine i resurse kako bi
poboljšale inovacijske rezultate i otkrile nove ideje \citep{asd3}.

\textbf{Intelektualno vlasništvo:}

Zatvorena inovacija može zadržati povjerljivost poslovnih tajni i zaštititi
intelektualno vlasništvo, dok otvorena inovacija uključuje dijeljenje znanja i
može uključivati visoke troškove za korištenje licenci i drugih oblika
intelektualnog vlasništva \citep{asd3}.

\textbf{Konkurencija:}

Zatvorena inovacija možda neće moći pratiti promjene u trendovima na tržištu i
konkurenciju, dok otvorena inovacija organizacijama omogućuje pristup raznolikim
znanjima i vještinama za razvoj inovativnih proizvoda ili usluga koji im mogu
pomoći ostati konkurentni \citep{asd4}.

Zaključno, otvorena inovacija i zatvorena inovacija su dva različita pristupa
upravljanju inovacijama, s različitim prednostima i nedostacima. Zatvorena
inovacija pruža kontrolu nad inovacijskim procesom i intelektualnim vlasništvom,
ali može rezultirati nedostatkom kreativnosti i inovacije zbog ograničenih
resursa i znanja. Otvorena inovacija organizacijama omogućuje pristup raznolikom
znanju i vještinama, ubrzava vrijeme dolaska na tržište i smanjuje troškove, ali
također nosi rizike poput procurivanja informacija i mogućnosti donošenja
pogrešnih odluka među startupima i tvrtkama koje nude inovativne proizvode i
tehnologije. Važno je procijeniti potrebe i resurse organizacije prije donošenja
odluke o odabiru zatvorene ili otvorene inovacijske strategije \citep{asd3}.