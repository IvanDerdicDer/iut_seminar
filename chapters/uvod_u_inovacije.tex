\chapter{Uvod u inovacije} \label{uvod_u_inovacije}

Inovacija je sposobnost koncipiranja, razvoja, isporuke i skaliranja novih
proizvoda, usluga, procesa i poslovnih modela za korisnike
\citep{mckinesyinnovation2022}. Točnije, inovacija je praktična implementacija
ideja koje rezultiraju uvođenjem novih dobara ili usluga ili poboljšanjem ponude
dobara ili usluga \citep{wikipediainnovation2023}. ISO TC 279 u standardu ISO
56000:2020 definira inovaciju kao "novu ili promijenjenu entitet koji ostvaruje
ili redistribuira vrijednost" \citep{wikipediainnovation2023}.

Inovacija se često ostvaruje razvojem učinkovitijih proizvoda, procesa, usluga,
tehnologija ili poslovnih modela koje inovatori stavljaju na raspolaganje
tržištima, vladama i društvu \citep{wikipediainnovation2023}. Inovacija je
povezana s, ali nije isto što i izum \citep{wikipediainnovation2023}. Inovacija
češće uključuje praktičnu primjenu izuma kako bi imala značajan utjecaj na
tržište ili društvo, i nije svaka inovacija povezana s novim izumom
\citep{wikipediainnovation2023}. Tehnička inovacija često se manifestira kroz
inženjerski proces kada se rješava tehnički ili znanstveni problem
\citep{wikipediainnovation2023}.

Inovacija može potjecati od različitih aktera, slučajno ili kao rezultat
značajnog kvara sustava \citep{wikipediainnovation2023}. Općenito, izvore
inovacija čine promjene u strukturi industrije, strukturi tržišta, lokalnoj i
globalnoj demografiji, ljudskoj percepciji, dostupnom znanstvenom znanju itd.
\citep{wikipediainnovation2023}. Tradicionalno prepoznati izvor inovacija je
inovacija proizvođača \citep{wikipediainnovation2023}. To je kada agent (osoba
ili tvrtka) inovira kako bi prodavao inovaciju. Drugi izvor inovacija je
inovacija korisnika \citep{wikipediainnovation2023}. To je kada agent (osoba ili
tvrtka) razvija inovaciju za vlastitu (osobnu ili internu) upotrebu jer
postojeći proizvodi ne zadovoljavaju njihove potrebe
\citep{wikipediainnovation2023}.

Postoje različite vrste inovacija, uključujući otvorenu inovaciju i inovaciju
korisnika \citep{wikipediainnovation2023}. Otvorena inovacija odnosi se na
korištenje osoba izvan organizacijskog konteksta koje nemaju stručnost u
određenom području kako bi riješile složene probleme
\citep{wikipediainnovation2023}. Inovacija korisnika je kada se tvrtke oslanjaju
na korisnike svojih dobara i usluga da osmisle, pomognu u razvoju i čak pomognu
u implementaciji novih ideja \citep{wikipediainnovation2023}.

Inovacija je višestupanjski proces u kojem organizacije pretvaraju ideje u
nove/unaprijeđene proizvode, usluge ili procese kako bi napredovale, natjecale
se i uspješno se diferencirale na svojem tržištu
\citep{wikipediainnovation2023}. Inovacija uključuje kombinaciju identifikacije
problema/prilika, uvođenja, usvajanja ili modificiranja novih ideja relevantnih
za organizacijske potrebe, promocije tih ideja i njihove praktične primjene
\citep{wikipediainnovation2023}.