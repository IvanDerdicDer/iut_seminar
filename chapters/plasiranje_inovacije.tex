\chapter{Plasiranje inovacije} \label{plasiranje_inovacije}

Prijedlog inovacije je alat za upravljanje SQL grupama, korisnicima i pravima.
Poterba za ovakvim alatom je nastala iz potrebe za jednostavnim upravljanjem
pravima na bazama podataka u više okolina. Upravljanje pravima na bazama
podataka je kompleksan posao koji zahtjeva znanje o bazama podataka i
operativnom sistemu. Trenutno ne postoji alat koji omogućuje jednostavnu
administraciju prava na bazama podataka. Upravljanje pravima na bazama podataka
se vrši kroz SQL upite koji su specifični za svaki SQL dijalekt. Prijedlog
inovacije omogućuje jednostavno upravljanje pravima na bazama podataka na način
da se centrealno definiraju prava i okoline, a zatim se prava dodjeljuju
korisnicima i grupama. Prijedlog inovacije omogućuje upravljanje pravima na
bazama podataka u više okolina, a prava se mogu definirati u konkretnom SQL
dijalektu. Inovacija nije namijenjena razvojnom osoblju već administratorima
baza podataka.

Trenutno ukoliko se želi upravljati pravima na bazama podataka u više okolina,
potrebno je za svaku okolinu ručno definirati prava. Ovakav pristup često može
dovesti do grešaka u definiranju prava. Također ovakav pristup je vremenski
zahtjevan i nije skalabilan. Uz to praćenje prava je teško ukoliko nisu dobro
dokumentirana. Predloženom inovacijom dokumentiranje bi bilo jednostavno jer bi
se prava definirala u jednom mjestu. Također bi se moglo pratiti koja prava su
dodjeljena korisnicima i grupama. Ukoliko bi se prava mijenjala, promjene bi se
mogle pratiti kroz verzioniranje.

\section{Pregled sličnih rješenja} \label{pregled_slicnih_rjesenja}

Trenutno na tržištu ne postoji alat koji omogućuje jednostavno upravljanje
pravima na bazama podataka kroz više okruženja. Trenutno tržište se fokusira na
razvojne alate koji omogućuju razvoj aplikacija. Posljedica toga je da alati za
upravljanje pravima na bazama podataka nisu razvijeni ili su zastarjeli. Postoji
alat Idera SQL Permission Extractor \citep{idera} koji omogućuje izvlačenj prava
iz postojeće baze podataka u SQL skriptu. Ovaj alat ne omogućuje upravljanje
pravima na bazama podataka. Također ne omogućuje upravljanje pravima na bazama
podataka kroz više okruženja. Također Idera SQL Permission Extractor radi samo
na Microsoft SQL Serveru. 

Postoji i alat SQL Server Permissions Manager \citep{ericcobb} koji omogućuje
upravljanje pravima na bazama podataka. Ovaj alat je set skripti koje se
izvršavaju bazi podataka. Skripte kreiraju zasebnu shemu i razne procedure koje
omogućuju verzioniranje, dodavanje i brisanje prava. Ovaj alat ne omogućuje
upravljanje pravima na bazama podataka kroz više okruženja. No ovaj alat može
raditi sa svim SQL bazama podataka na jednom SQL serveru.

\section{Tehnički opis inovacije} \label{tehnicki_opis_inovacije}

Prijedlog inovacije je CLI aplikacija koja omogućuje upravljanje pravima na
bazama podataka. Aplikacija je napisana u programskom jeziku Python. Aplikacija
koristi ODBC ili JDBC konekciju za komunikaciju sa bazom podataka. ODBC i JDBC
su standardi za komunikaciju sa bazama podataka. Ovi standardi omogućuju
komunikaciju sa raznim bazama podataka. Aplikacija je zamišljena da podržava sve
SQL dijalekte. To se postiže tako da se prava definiraju u SQL upitima, a ne
kroz parametre. Također aplikacija podržava verzioniranje prava. Verzioniranje
prava se postiže tako da se prava definiraju u datotekama. Datoteke se nalaze u
direktoriju koji je inicijaliziran kao git repozitorij. Također aplikacija
omogućuje upravljanje pravima kroz više okruženja. Ovo se postiže tako da se u
konfiguracijsku datoteku aplikacije definiraju okruženja. U konfiguracijskoj
datoteci se definiraju konekcije za svako okruženje. Skripte koje definiraju se
mogu parametrizirati za izvršavanje na više okruženja korištenjem Jinja2
templating jezika. Jinja2 je templating jezik koji omogućuje parametrizaciju
teksta. Sličan pristup koristi alat DBT \citep{dbt} koji omogućuje upravljanje
tablicama na bazama podataka. Pošto alat za upravljanje pravima izvršava SQL
skripte na bazama podataka nije potrebno jako računalo za pokretanje alata.
Alat će biti dostupan kao CLI aplikacija. CLI aplikacija će biti dostupna za
Linux, Windows i Mac operativne sustave. CLI aplikacija će biti dostupna kao
Python paket. Python paket će biti dostupan na Python paket menadžerima PyPI i
Conda. 

\section{Prednosti inovacije} \label{prednosti_inovacije}

Ova inovacija omogućuje jednostavno upravljanje pravima na bazama podataka kroz
više okruženja. Također omogućuje verzioniranje prava. Centralizacijom prava
postiže se jednostavnije upravljanje pravima, jednostavnije praćenje prava i
jednostavnije dokumentiranje prava. Korištenjem GITa za verzioniranje prava
postiže se jednostavnije praćenje promjena prava i jednostavno vraćanje na
prethodno stanje. Također korištenjem GITa za verzioniranje prava postiže se
jednostavnije praćenje tko i kada je napravio promjenu prava. Korištenjem GITa
za verzioniranje prava postiže se jednostavnije praćenje zašto je napravljena
promjena prava. Pošto alat sam izvršava SQL skripte na bazama podataka smanjuje
se mogućnost ljudske pogreške. Korištenjem alata smanjuje se vrijeme potrebno za
upravljanje pravima. Korištenjem alata smanjuje se vrijeme potrebno za
upogonjenje beze podataka na novoj okolini i time se ubrzava proces razvoja.
Korištenjem alata smanjuje se vrijeme potrebno za dokumentiranje prava. Ukoliko
dođe do katastrofalnog gubitka baze podataka, alat omogućuje brzo ponovno
uspostavljanje prava na bazi podataka.

\section{Ciljna skupina} \label{ciljna_skupina}

Ciljna skupina su tvrtke koje razvijaju aplikacije koje koriste SQL baze
podataka. Također ciljna skupina su tvrtke koje imaju više okruženja za
razvoj aplikacija. Također ciljna skupina su tvrtke koje imaju više razvojnih
timova koji rade na istoj bazi podataka. Unutar tvrtki ciljna skupina su
administratori baza podataka, razvojni inženjeri i DevOps inženjeri.
Također cilja skupina su pojedinci koji dovoljno poznavaju SQL i žele
jednostavnije upravljati pravima na bazama podataka. No glavna ciljna skupina su
tvrtke koje se bave podacima. Specifično tvrtke koje se bave analitikom podataka
i tvrtke koje se bave upravljanjem podacima. Takvim tvrtka je bitno da imaju
robustan sustav za upravljanje pravima na bazama podataka. Takvim tvrtkama je
bitno da imaju centralizirano upravljanje pravima na bazama podataka. Takvim
tvrtkama je bitno da imaju verzioniranje prava na bazama podataka. Takvim
tvrtkama je bitno da imaju dokumentiranje prava na bazama podataka.

\section{Konkurencija} \label{konkurencija}

Trenutno na tržištiu ne postoji direktna konkurencija. Kako je 
navedeno u poglavlju~(\ref{pregled_slicnih_rjesenja}) postoje alati koji
implementiraju neke segmente ove inovacije. No niti jedan alat ne implementira
sve segmente ove inovacije. Također niti jedan alat ne implementira segmente
ove inovacije na način na koji je to implementirano u ovoj inovaciji. 

\section{Komerijalizacija} \label{komerijalizacija}

Alat će biti dostupan kao program otvorenog koda. Alat će biti dostupan na
GitHubu. Alat će biti dostupan pod MIT licencom. Nuuditi će se podrška za
alat. Podrška će se nuditi kroz održavanje alata i kroz konzultacije. Održavanje
i konzultacije će se naplaćivati po satu. Održavanje i konzultacije će se
naplaćivati u eurima. Održavanje i konzultacije će se naplaćivati prema
cjeniku. 
